\documentclass[uplatex,dvipdfmx]{jsarticle}

\usepackage[uplatex,deluxe]{otf} % UTF
\usepackage[noalphabet]{pxchfon} % must be after otf package
\usepackage{stix2} %欧文&数式フォント
\usepackage[fleqn,tbtags]{mathtools} % 数式関連 (w/ amsmath)
\usepackage{hira-stix} % ヒラギノフォント&STIX2 フォント代替定義(Warning回避)


\usepackage{ascmac}
\usepackage{url}
\usepackage{float}
\usepackage{moreverb}
\usepackage{lscape}

\begin{document}

\title{モビリティレポート}
\author{25G1065 塩澤匠生}
\date{2025年6月11日}
\maketitle
\section{はじめに}

今回のレポートはアイデアソンのレポートの一環として
モビリティの問題点や改善点を考え,
IT(IoT,ICTを含む)を用いて5年以内に解決・改善できる手法を考案する


\subsection{社会的背景}

社会的背景として自転車の単独事故は年5497件発生していて,その割合は年々増加している
というものがある\cite{jikokensuu}.
単独事故の割合は,年々増加しているというものがある.そして,
単独事故の原因の内7割が転倒事故であるというものがある\cite{tandokuWariai}.

\subsection{問題点}

ここで,車の単独事故の原因の内訳と比較してみると車の単独事故の内転倒(横転)事故は1\%
であることから,自転車は乗り物の性質上転倒事故が発生するという問題があると考えることができる.



%一般的な社会的背景を記述したパラグラフを複数配置, パラグラフの終わりは\parを入れる。
%最終パラグラフにはアイデアを記述する.

こんな感じで文献引用する\cite{ref:nobukawa2023,ref:nobukawa2023_2}.

\subsection{目的}
こんな感じで文献引用する\cite{ref:nobukawa2023,ref:nobukawa2023_2}.

\section{解決策としての提案手法}
%アイデアを実現する概要図を入れてその内容を説明する.
必要に合わせて,subsectionにわけて図を図\ref{fig:problem}や表\ref{table:presentation}
を引用しつつ説明する.
自転車の性質上,転倒事故が発生しやすいという問題に対して我々は自転車姿勢制御モジュールというものを
提案する.自転車姿勢制御モジュールの概念図を以下図\ref{fig:moduleGainenn}に示す

\begin{figure}[H]
    \centering
    \includegraphics[width=0.8\textwidth]{fig/moduleGainenn.png}
    \caption{自転車姿勢制御モジュールの概念図}
    \label{fig:moduleGainenn}
\end{figure}

まず,先行研究として,JAXAが開発した超小型三軸姿勢制御モジュールというものがある.
このモジュールは人工衛星の小型化を図るために作られたものでサイズは$10×10×10 {cm}^3$である.





\section{提案手法の実現可能性の評価と妥当性の検証}
%

\section{おわりに}
%全体のまとめを簡潔に記述して,結論を述べる.




\begin{thebibliography}{9}

\bibitem{jikokensuu} NEONAVI, 【自転車事故の実態】を知って安全に利用しよう~令和5年「交通事故統計」から, 
2025年6月11日閲覧,\url{https://neonavi.info/11203/}

\bibitem{tandokuWariai} 東京海上日動, 便利な自転車は運転次第で危険な乗り物になる, 
2025年6月11日閲覧,\url{https://www.tokiomarine-nichido.co.jp/world/guide/drive/202105.html}

\bibitem{ref:nobukawa2023} 礒川悌次郎, \& 信川創. (2023). 脳・神経系における機能創発の解明を目指した数理モデリングとデータ駆動分析―局所神経回路から大域的全脳レベルまで―. 計測と制御, 62(10), 587-592.

\bibitem{ref:nobukawa2023_2}  信川創. "ヒトの認知行動を推定するアイマーカーの脳内メカニズム." 体育の科学, 73.10 (2023): 658-662.
\end{thebibliography}

\end{document} 
