\documentclass[uplatex]{jsarticle}
\usepackage{amsmath}
\usepackage[dvipdfmx]{graphicx}

\setcounter{tocdepth}{3}
\usepackage{float}
\usepackage{moreverb}
\usepackage{lscape}
%\pagestyle{empty}
%\usepackage{wrapfig}
%\usepackage{url}
%\usepackage{EasyLayout}

\usepackage{ascmac}
%\usepackage{fancybx}

%\pagestyle{myheadings}



\begin{document}

\begin{flushright}
    \number\year 年\number\month 月\number\day 日
\end{flushright}

\begin{center}
    {\LARGE アイディアソンポスターセッション報告書}
\end{center}

\begin{flushright}
    グループ番号:01\\
    学生番号:25G1065\\
    氏名:塩澤匠生
\end{flushright}


\section{背景・問題点・解決案}
% 2〜3行程度の文章で説明する.もう少し長くても可とする.
この班では社会的背景として,自転車を利用している人が多く,交通事故が多いことを挙げた.問題点として背景として
のような状況があるため安全運転を心がけたいが,自転車の性質上スピードを落とすと姿勢が不安定になる点や,段差などによる
回避の難しいのない事故の発生があるという問題点がある.そこで解決策として,自転車の姿勢制御モジュールを提案した.


\section{質問と回答}
% 質問された内容と,それに対する理想的な回答や対応を箇条書きで記述する.似たような内容に対してまとめて回答を書いても良い.
% 以下は例なので,提出前に消すこと.
% 質問毎に1行空けて,文章上で段落を分けること.
自転車をカーブさせるときに姿勢制御モジュールが働いてしまって,車体を傾けてカーブさせることができなくなるのではないかという
質問を受けた.回答として,車問を傾けてカーブさせられるということは,十分車体のスピードが出ていて姿勢制御を
行う必要がないため,姿勢制御を無効にするという回答をした.

モジュールはどこにつけるのかという質問を受けた.ママチャリの後ろの荷台につける想定だという回答をした.

車相手の急な事故の時避けられないのではないかという質問を受けた.回答として,このモジュール自体,車との事故を防ぐ目的ではなく,歩道などを低速で運転することで
歩行者との接触事故を減らしたり,単独での転倒事故を減らすことを目的としているという説明をした.

制御できるほどパワーがあるのかというという質問を受けた.回答として,先行研究のモーターでは厳しいかもしれないが,技術を応用しているだけなので,
大きいモーターやブラシレスモーターを使用することで十分なパワーを得られるという回答をした.

何km/hから徐行?何度の傾きからが補正対象?という質問を受けた.実際に実験しながらデータを集め,それ基準で選ぶようにし,
また,使う人の側でも調整できるようにするシステムにするという回答をした.





\section{他のグループで参考になった点}
% 問題点に対する解決案が良かった,ポスターの作り方が良かった,説明が丁寧だったなど,今後の参考になるグループが有ったらグループ番号とその内容を書く.
% 以下は例なので,提出前に消すこと.
% 項目ごとに1行空けて,文章上で段落を分けること.
グループ02:ポスターが見やすく,使用する色の種類が少なく,また,使用する色が見やすいものになっていてとても良かった.次からの参考にしたい.

グループ02:提案するシステムが他の班より考え込まれており,説明とスライドの内容でしっかりとイメージすることができてよかった.



\end{document}



























