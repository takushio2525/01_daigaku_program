\documentclass[uplatex]{jsarticle}
\usepackage{amsmath}
\usepackage[dvipdfmx]{graphicx}

\setcounter{tocdepth}{3}
\usepackage{float}
\usepackage{moreverb}
\usepackage{lscape}
%\pagestyle{empty}
%\usepackage{wrapfig}
%\usepackage{url}
%\usepackage{EasyLayout}

\usepackage{ascmac}
%\usepackage{fancybx}

%\pagestyle{myheadings}



\begin{document}

\begin{flushright}
    \number\year 年\number\month 月\number\day 日
\end{flushright}

\begin{center}
    {\LARGE アイディアソンポスターセッション報告書}
\end{center}

\begin{flushright}
    グループ番号:00\\
    学生番号:25G1999\\
    氏名:工大太郎
\end{flushright}


\section{背景・問題点・解決案}
% 2〜3行程度の文章で説明する.もう少し長くても可とする.


\section{質問と回答}
% 質問された内容と,それに対する理想的な回答や対応を箇条書きで記述する.似たような内容に対してまとめて回答を書いても良い.
% 以下は例なので,提出前に消すこと.
% 質問毎に1行空けて,文章上で段落を分けること.
ポスターの背景部分について,どこからデータを得たのか,またそのデータの信頼性について質問を受けた.
データのソースを示す必要があるので,次回はしっかり記述する.
また,信頼性については公的機関が出しているデータなので信頼性が高いことを回答する.

解決案について,実現できそうか質問を受けた.
これから考えていくので,現状では回答できない.

\section{他のグループで参考になった点}
% 問題点に対する解決案が良かった,ポスターの作り方が良かった,説明が丁寧だったなど,今後の参考になるグループが有ったらグループ番号とその内容を書く.
% 以下は例なので,提出前に消すこと.
% 項目ごとに1行空けて,文章上で段落を分けること.
グループ99:ポスターのレイアウトが良く,全体の流れが分かりやすかった.

グループ99:問題点の着眼点が良かった.

\end{document}



























