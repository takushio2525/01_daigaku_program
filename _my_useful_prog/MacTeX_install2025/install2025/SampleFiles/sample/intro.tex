\section{はじめに}

{\TeX}(テフあるいはテックと読む)とは,コンピュータ科学者
Donald E.\ Knuth \cite{Knuth}が1978年に公開を始めた文書整形処理システムである.
印刷業界の用語では,\ruby{{\gtbf{組版}}}{くみはん}処理システムとも呼ばれ,用意した原稿素材
(テキスト・図版・写真等)を各言語のルール\cite{JIS}\cite{typst}の下に,指定のレイアウトに
なるように配置するソフトウェアである.
{\TeX}を用いると,OS(Operating System)に依存せずに出力結果の見た目を統一でき,特に数式の
仕上がりが綺麗なため\footnote{組版専用ソフトの代表的なものにAdobeのInDesignがあるが,数式の
組版は{\TeX}には敵わない.また,Microsoft Wordなどのワープロソフトでは,まともな組版処理は
できない.同じレイアウトにするのに,InDesignや{\TeX}に比べて遥かに面倒なだけでなく,実現できない
処理も多いからである.例えば,Unicode対応フォントには,一つのコードポイントで,旧字体やルビ字体
など複数のデザインが含まれている文字をもつものがあり(名前にProNやPr6Nがつくフォントが相当),
それらの字体はGlyph ID(GID)やCharacter ID(CID)で区別するが,WordはGID/CIDを扱えない.
また,商業印刷で使われるプロダクションプリンタは色をCMYK(Cyan, Magenta, Yellow, and Key)で
出力するが,WordはRGB(Red, Green, and Blue)しか扱えないため色味が変わってしまうなど,
制約が多すぎる.},科学技術の分野では多くの出版物で利用されている.
\begin{figure}[b]
\begin{itembox}[l]{\gtbf{Version} \bfseries{\robotolgr p}}
\small\sffamily\mgfamily
Knuth博士は,1978年の初版の公開後も{\TeX}の改良や拡張を行ってきたが,1989年の
バージョン3の発表時に,これ以上の機能拡張は行わず以降は不具合の修正のみを行っていく
ことを宣言した.バージョン番号は,3.1415$\cdots$と更新のたびに円周率に近づけることに
なっており,Knuth博士の死没をもってバージョン{\robotolgr p}として更新が終了することになっている.
2025年4月時点でのバージョンは3.141592653(2021年2月5日)である.
\end{itembox}
\end{figure}

情報工学科では,BYOD(Bring Your Own Device)の機種としてAppleのMacBook Air/Proを
指定しており,課題等の文書作成には{\TeX}を使うことを原則としている.
そのため,学科独自の設定を盛り込んだ,{\TeX}の設定スクリプトを用意しているが,
本稿は提出物作成の際に使用する標準的な書式のテンプレートを用いたサンプルであり,
学科独自の設定の概略を説明するものである.
{\TeX}の使い方そのものを説明する文書ではなく,あくまでテンプレート代わりの
サンプルとして用意したものである.とはいえ,数多くのコマンドを意図的に使っているため,
様々な場面で参考になるはずである.是非,活用して欲しい.
なお,添付している本稿のソースファイルでは,高度な設定を必要とするものを
省略しているため,コンパイルしてもこのPDF(Portable Document Format)\cite{PDF}ファイルと同一の
見た目にはならないことを予めお断りしておく.

{\TeX}は,HTML(HyperText Markup Language)のようなマークアップ言語の一種である.
したがって,そのソースファイル(拡張子は\texttt{.tex})は,文章そのものと文章の構造や
見た目を指定するコマンドから成るテキストファイルである.複雑な数式や記号もテキストで入力する\footnote{テキストによる数式の入力方式は多くのソフトウェア(Microsoft Word/Excel/PowerPoint, Apple Pages/Numbers/ Keynoteなど)に
取り入れられており,事実上のスタンダードになっている.}.
例えば,ギリシャ文字の$\pi$は,「ぱい」を変換してπとする(和文フォントが使われてしまう)のではなく,\verb|\pi|と入力する.単なるテキストファイルであるためOSに依存せず作成・編集でき,コンパイルすることによりファイル中のコマンドに基づいて文書が組版される.
組版結果はDVI(\textbf{\textsf{d}}e\textbf{\textsf{v}}ice-\textbf{\textsf{i}}ndependent)形式のファイル(拡張子は \texttt{.dvi})に書き出される.
DVIファイルは,表示デバイスやプリンタなどの装置に依存しない中間形式のバイナリデータであり,
DVIドライバと呼ばれる別のソフトウェア(DVIウェアとも言う)で組版結果をプレビューしたり,印刷可能なPostScript\footnote{Adobe社が開発したページ記述言語であり,文字や画像をベクトルデータ(テキスト)で記述するため,高解像度の印刷が可能である.}\cite{PS}ファイルに変換したりして利用する.また,近年ではDVIファイルをPDF\footnote{Adobe社が開発した電子文書のファイル形式で,PostScriptをベースにしているため,拡大・縮小しても画像や文字が粗くならない.ただし,写真はピクセル単位で表現されるラスタ形式のため,拡大すると粗くなる.}に変換して,PDFファイルを最終出力とするのが一般的である.

{\TeX}はオープンソースソフトウェア\footnote{ソースコードが公開されており,無償で使用でき,誰でも自由に修正,改変,再配布が可能である.}であるため,組版処理を行うエンジンには,いくつもの派生系が存在している.中でも,複雑になりがちな各種の設定をマクロファイル(クラスファイルとパッケージファイルがある)を読み込むことで簡易に行える{\LaTeX} \cite{latex}がLeslie B.\ Lamport によって開発されて以降は,{\TeX}と言えば{\LaTeX}を指していることが普通である.
ただし,{\LaTeX}にも多くの派生エンジンが存在し,日本では縦書きや禁則処理などの日本語固有の処理を扱えるようにした{p\LaTeX}\footnote{出版社の株式会社アスキー(現在は株式会社角川アスキー総合研究所)が自社の出版物の組版をするために日本語化した.{p\LaTeX}のpはpublishingの意味である.}\cite{ptex},
さらに近年のOSで主流となったUnicode対応フォント(OpenTypeフォント)\cite{unicode}\cite{opntyp}を扱えるようにした{up\LaTeX} \cite{uptex}が長いこと主流である.世界的には,Unicode対応フォントを柔軟に扱え,かつ組版結果を直接PDFに出力できる{Lua\LaTeX} \cite{luatex}が主流になりつつあり,日本語を扱える{Lua\TeX-ja} \cite{luatexj}も日々進化している.ただし,{up\LaTeX}と{Lua\TeX-ja}はコマンド体系に違いがあるため,{Lua\TeX-ja}では{up\LaTeX}のソースコードをそのままコンパイルできない.
将来的には日本でも{Lua\LaTeX}が主流になることが予想されているが,現状では出版社や学術団体が用意しているマクロファイルの多くが{(u)p\LaTeXe}に基づいているため,本稿でも\gtbf{{up\LaTeX}の使用を前提}として説明をしていく.独学で学ぶ意欲のある方は,最初から{Lua\TeX-ja}を使っていくのも良いであろう(インストールはされている).ただし,学科として設定を統一したり,テンプレートを配布するのは,仕上がりの文書の体裁を統一するためでもあるので,そのことには注意を払うべきである.

\begin{figure}[bt]
\begin{itembox}[l]{\gtbf{{\TeX}, {\LaTeX}, {\LaTeXe}}}
\small\sffamily\mgfamily
{\TeX}はソースコードが公開されており,誰でも改良を加えることができる.また,オリジナルの{\TeX}と区別できる名前を付けさえすればその改良版の配布も許されているため,数多くの派生エンジンが存在している.{\LaTeX}の初期のバージョンは{\LaTeX} 2.09と呼ばれていたが,この派生系はそれぞれ独自に拡張されていったため互換性がなく,ソースファイルを見ても,どの派生{\LaTeX}エンジンでコンパイル可能なのかの判別が難しかった.そこで,それらの拡張をすべて網羅するようコマンド体系を整理した{\LaTeXe}が開発された.現在の派生エンジンの多くは{\LaTeXe}が元になっているが,再び互換性の問題が起きつつある.また,次期バージョンとして{\LaTeX} 3のプロジェクトが進行中である.
\end{itembox}
\end{figure}
