\documentclass[uplatex]{jsarticle}
\usepackage{amsmath}
\usepackage[dvipdfmx]{graphicx}

\setcounter{tocdepth}{3}
\usepackage{float}
\usepackage{moreverb}
\usepackage{lscape}
%\pagestyle{empty}
%\usepackage{wrapfig}
%\usepackage{url}
%\usepackage{EasyLayout}

\usepackage{ascmac}
%\usepackage{fancybx}

%\pagestyle{myheadings}



\begin{document}

\begin{flushright}
    \number\year 年\number\month 月\number\day 日
\end{flushright}

\begin{center}
    {\LARGE 課題8}
\end{center}

\begin{flushright}
    学生番号:25G1065\\
    氏名:塩澤匠生
\end{flushright}


\section{選択ソート}
課題6でおこなったようなソート方法。時間計算量が入力されるリストの量によって大きくなる。この処理方法
ではどんな値が入力されても絶対に回数分計算が行われる。このアルゴリズムを使うためにiとかのループ以外に
ほとんど新しく変数を確保する必要はないので空間計算量はとても小さい。
\section{クイックソート}
クイックソートとは入力された配列に対して基準の値を一つ決め、その値以上か以下かの2つの配列に分け、そのその配列に対してまた同様の処理を
行っていくものである。
基準にするデータによって時間計算量も空間計算量も変わってくる。時間計算量は上下するが
平均の回数は選択ソートに比べて短くなる。空間計算量は選択ソートと違って分割した配列分のメモリが必要なので
選択ソートより多い。

\section{マージソート}
マージソートは配列を一つずつの要素に分割して比較しながら統合していく方法。
時間計算量が一定かつ効率がいいが、大きく空間計算量を要求される。

\section{バブルソート}
隣と比べて値が大きいか小さいかを繰り返し比較しながら並び替えていく方法。
追加のメモリがかからないため空間計算量は小さいが時間計算量が大きくなる。





\end{document}



























