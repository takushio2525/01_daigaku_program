\documentclass[uplatex]{jsarticle}
\usepackage{amsmath}
\usepackage[dvipdfmx]{graphicx}

\setcounter{tocdepth}{3}
\usepackage{float}
\usepackage{moreverb}
\usepackage{lscape}
%\pagestyle{empty}
%\usepackage{wrapfig}
%\usepackage{url}
%\usepackage{EasyLayout}

\usepackage{ascmac}
%\usepackage{fancybx}

%\pagestyle{myheadings}



\begin{document}

\begin{flushright}
    \number\year 年\number\month 月\number\day 日
\end{flushright}

\begin{center}
    {\LARGE 課題7}
\end{center}

\begin{flushright}
    学生番号:25G1065\\
    氏名:塩澤匠生
\end{flushright}


\section{時間計算量}
アルゴリズムを実行するのにかかる時間のこと。今回のmymax関数なら入力するリストの大きさが大きくなるほど
長くなる
\section{空間計算量}
空間計算量とは引数とするデータとは別に、アルゴリズムを実行するために使うメモリ量のとこ。今回のmymax関数なら引数として渡しているリストは
別の場所で宣言されているためこれに含まれない。判定に使っているmxvのことを言う。
\end{document}



























