\section{フォントについて}

前節冒頭で「{\TeX}を使うとOSに依存せずに出力結果の見た目を統一できる」と書いたが,それには前提条件があり,出力するコンピュータに同じフォントがインストールされている場合に限定される.
予定されたフォントがインストールされておらず別のフォントが代用されると,文字幅等が変わるのでページレイアウト自体は同一でも文章の行数が変わったり,見た目の印象も変わったりする.
情報工学科が推奨するMac{\TeX}\cite{mactex}のインストーラを用いると,オープンソースの無償フォントが数多くインストールされる.デフォルトでは,欧文にKnuth博士がデザインしたComputer Modern(CM)フォント\cite{cm},和文に原ノ味フォント\cite{harano}が使用される.
\begin{figure}[b]
\begin{itembox}[l]{\gtbf{Mac{\TeX}}}
\small\sffamily\mgfamily
世界的に最も普及している{\TeX}のディストリビューションは{\TeX} Liveであり,多くのOSプラットフォームで利用できる.Mac{\TeX}は,macOSに特化した{\TeX} Liveのインストーラである.一方,{\TeX} Liveは,Homebrew \cite{homebrew}やMacPorts \cite{macports}といった
パッケージ管理ツールを用いてインストールすることも可能である.しかし,両者はインストール先のディレクトリが異なるため,混在させるとファイルの依存関係が崩れ,正しく動かない実行ファイルがでてきたりする.したがって,Mac{\TeX}でインストールを行ったのであれば,以降も{\TeX}の更新はMac{\TeX}を用いるか,移行するのであれば,PATHの設定を変更する必要がある.
\end{itembox}
\end{figure}

フォントは好みもあるが,一般に有償の和文フォントは,様々な状況において見た目のバランスに優れ,出力結果の品質が高いのに対し,無償の和文フォントは品質のばらつきが大きい(和文には数万文字を超える漢字があるため,文字数の少ない欧文フォントに比べて,デザイン調整に膨大な手間が必要となるからである).macOSの日本語環境で使用されるヒラギノフォントは,癖のない高品質な有償フォントであるため,多くの出版物に採用\footnote{高速道路の標識にもヒラギノ角ゴシックが採用されている.}されている.そこで,情報工学科の用意した{\TeX}の設定スクリプトでは\gtbf{和文フォントにヒラギノを使用}する設定にしている.

ところで,現在のレーザープリンタはPostScript(PS)対応あるいはPS互換であることが一般的である.PSプリンタは,フォントもベクトルデータとして処理する\footnote{インクジェットプリンタでは,インク滴が1つの点(ドット)となり,その点を連続して吹きつけて面を構成することで,文字やグラフィックを表現している.}ため,拡大・縮小印刷をしても綺麗に印刷される.また,よく使われるフォントはプリンタに内蔵しておくと印刷が速い(そうでない場合は,フォントデータを都度プリンタに送信する必要がある).初期のPostScript Level 1(PS1)プリンタ\footnote{世界(そして日本)初のPSプリンタはAppleのLaserWriterである.}では,欧文フォントにセリフ体のTimes,サンセリフ体のHelvetica,タイプライタ体のCourier Newのそれぞれでローマン体,イタリック体,ボールド体,ボールドイタリック体,加えて記号用のSymbolおよびZapf Dingbatsの基本14書体(Base 14 fonts)が搭載されていた\footnote{PostScript Level 2(PS2)プリンタでは,基本14書体に加え,セリフ体としてPalatino,Bookman,New Century Schoolbook,サンセリフ体としてHelvetica Narrow,Avant Gardeそれぞれのローマン体,イタリック体,ボールド体,ボールドイタリック体と,筆記体のZapf Chanceryを加えた基本35書体(Base 35 fonts)が搭載された.現在の主流は,PostScript 3プリンタ(Levelをつけない)であり136書体の欧文フォントが搭載されている.}.
これらの基本14書体はAcrobat Readerにも内蔵されている(フォントが埋め込まれていないPDFで代用フォントとして用いられる\footnote{最近のAcrobat Readerでは,HelveticaとTimesが,それぞれArialとTimes New Romanに置き換えられている.}).

一方,日本語に初対応したPostScript Level 2(PS2)プリンタでは,欧文のセリフ体とサンセリフ体に相当する明朝体とゴシック体(モリサワのリュウミンL-KLと中ゴシックBBB)の基本2書体が搭載された.その後,明朝体とゴシック体が2ウエイト(太ミンA101と太ゴB101)が追加になり,さらに丸ゴシック体(じゅん101)も搭載され,基本5書体と呼ばれた.これらのフォントは当初パッケージとしては販売されておらず,プリンタとフォントを一体購入し,パソコンの画面上では解像度の粗いスクリーンフォント\footnote{ビットマップフォントと呼ばれ,各文字をピクセルのオン・オフで表現する.一方,文字の輪郭を曲線(ベクトル)で表現するPSフォントは,アウトラインフォントと呼ばれる.}を使用し,印刷すると綺麗な仕上がりになるという使用法\footnote{パソコンの処理速度が遅く,アウトラインフォントを画面に直接描画できない事情もあった.}であった.
このような背景から,日本語を扱う(u)p{\LaTeX}でも標準では丸ゴシック体を除く4書体が使用されてきた(印刷時はプリンタ内蔵のフォントが使われた).現在主流のPostScript 3プリンタに搭載される和文フォントは,平成明朝W3と平成角ゴシックW5であることが多い.しかし,近年のパソコンは処理能力が飛躍的に向上し,画面上でもアウトラインフォントを直接描画できるようになったため,プリンタの搭載フォントによらず任意のフォントを使用して書類を作成し,印刷することが可能である.ただし,プリンタに搭載されていないフォントを使用すると印刷にかかる時間は増える.

\begin{figure}[b]
\begin{itembox}[l]{\gtbf{ウエイト}}
\small\sffamily\mgfamily
フォントの太さはウエイト(Weight)で表される.フォントベンダーにより太さの基準や呼び方に違いはあるが最大で10段階程度に分類される.ISOでは9段階に分類されており,それぞれ,W1: Ultra Light(極細,Thin),W2: Extra Light(特細,Ultra Light),W3: Light(細),W4: Semi Light(中細,Regular/Normal),W5: Medium(中),W6: Semi Bold(中太,Demi Bold),W7: Bold(太),W8: Extra Bold(特太,Ultra Bold),W9: Ultra Bold(極太,Black/Heavy)と呼ばれる.カッコ内は日本語やISO以外での慣例的な呼び方である.ただし,実際にはこの表記が当てはまらないフォント製品も数多くある.
\end{itembox}
\end{figure}
