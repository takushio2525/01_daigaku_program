\section{はじめに}

{\TeX}(テフあるいはテックと読む)とは,コンピュータ科学者
Donald E.\ Knuth \cite{Knuth}が1978年に公開を始めた文書整形処理システムである.
印刷業界の用語では,組版処理システムとも呼ばれ,
用意した原稿素材(テキスト・図版・写真等)を各言語のルール\cite{JIS}\cite{typst}の下に,
指定のレイアウトになるように配置するソフトウェアである.
{\TeX}を用いると,OS(Operating System)に依存せずに出力結果の見た目を統一でき,
特に数式の仕上がりが綺麗なため,科学技術の分野では多くの出版物で利用されている.

情報工学科では,BYOD(Bring Your Own Device)の機種としてAppleのMacBook Air/Proを
指定しており,課題等の文書作成には{\TeX}を使うことを原則としている.
そのため,学科独自の設定を盛り込んだ,{\TeX}の設定スクリプトを用意しているが,
本稿は提出物作成の際に使用する標準的な書式のテンプレートを用いたサンプルであり,
学科独自の設定の概略を説明するものである.
{\TeX}の使い方そのものを説明する文書ではなく,あくまでテンプレート代わりの
サンプルとして用意したものである.とはいえ,数多くのコマンドを意図的に使っているため,
様々な場面で参考になるはずである.是非,活用して欲しい.
なお,添付している本稿のソースファイルでは,高度な設定を必要とするものを
省略しているため,コンパイルしてもこのPDF(Portable Document Format)ファイルと同一の見た目にはならいないことを
予めお断りしておく.

{\TeX}は,HTML(HyperText Markup Language)のようなマークアップ言語の一種である.
したがって,そのソースファイル(拡張子は\texttt{.tex})は,文章そのものと文章の構造や
見た目を指定するコマンドから成るテキストファイルである.複雑な数式や記号もテキストで入力する.
例えば,ギリシャ文字の$\pi$は,「ぱい」を変換してπとする(和文フォントが使われてしまう)のではなく,\verb|\pi|と入力する.単なるテキストファイルであるためOSに依存せず作成・編集でき,コンパイルすることによりファイル中のコマンドに基づいて文書が組版される.
組版結果はDVI(\textbf{\textsf{d}}e\textbf{\textsf{v}}ice-\textbf{\textsf{i}}ndependent)形式のファイル(拡張子は \texttt{.dvi})に書き出される.
DVIファイルは,表示デバイスやプリンタなどの装置に依存しない中間形式のバイナリデータであり,
DVIドライバと呼ばれる別のソフトウェア(DVIウェアとも言う)で組版結果をプレビューしたり,印刷可能なPostScript\cite{PS}ファイルに変換したりして利用する.また,近年ではDVIファイルをPDF\cite{PDF}に変換して,PDFファイルを最終出力とするのが一般的である.

{\TeX}はオープンソースソフトウェアであるため,組版処理を行うエンジンには,いくつもの派生系が存在している.中でも,複雑になりがちな各種の設定をマクロファイル(クラスファイルとパッケージファイルがある)を読み込むことで簡易に行える{\LaTeX} \cite{latex}がLeslie B.\ Lamport によって開発されて以降は,{\TeX}と言えば{\LaTeX}を指していることが普通である.
ただし,{\LaTeX}にも多くの派生エンジンが存在し,
日本では,縦書きや禁則処理などの日本語固有の処理を扱えるようにした{p\LaTeX}\cite{ptex},
さらに近年のOSで主流となったUnicode対応フォント(OpenTypeフォント)\cite{unicode}\cite{opntyp}を扱えるようにした{up\LaTeX} \cite{uptex}が長いこと主流である.世界的には,Unicode対応フォントを柔軟に扱え,かつ組版結果を直接PDFに出力できる{Lua\LaTeX} \cite{luatex}が主流になりつつあり,日本語を扱える{Lua\TeX-ja} \cite{luatexj}も日々進化している.ただし,{up\LaTeX}と{Lua\TeX-ja}はコマンド体系に違いがあるため,{Lua\TeX-ja}では{up\LaTeX}のソースコードをそのままコンパイルできない.
将来的には日本でも{Lua\LaTeX}が主流になることが予想されているが,現状では出版社や学術団体が用意しているマクロファイルの多くが{(u)p\LaTeXe}に基づいているため,本稿でも{\gtbf {up\LaTeX}の使用を前提}として説明をしていく.独学で学ぶ意欲のある方は,最初から{Lua\TeX-ja}を使っていくのも良いであろう(インストールはされている).ただし,学科として設定を統一したり,テンプレートを配布するのは,仕上がりの文書の体裁を統一するためでもあるので,そのことには注意を払うべきである.

