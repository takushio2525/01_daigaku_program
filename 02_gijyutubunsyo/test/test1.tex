\documentclass[uplatex]{jsarticle}

\usepackage{amsmath}
\usepackage[dvipdfmx]{graphicx}
\setcounter{tocdepth}{3}
\usepackage{float}
\usepackage{moreverb}
\usepackage{lscape}
\usepackage{ascmac}

\title{第6回目の課題}
\author{25G1065 塩澤匠生}
%\date{2015年11月13日}

\begin{document}

\maketitle

\section{演習問題:行内に数式を記述する場合}
% この下の行に入力

オームの法則は,電圧$V$[V] と電流$I$[A],抵抗$R$ [$\Omega$] の間に成立する関係: $V=IR$で表すことができる.

\section{演習問題: 独立した行に数式を記述する場合}
% この下の行に入力

オームの法則は,電圧$V$[V] と電流$I$[A],抵抗$R$ [$\Omega$] の間に成立する関係式で
(\ref{eq:VIR}) 式で表すことができる.\\
\begin{equation}
V=IR\label{eq:VIR}
\end{equation}
ここで,本実験では,$R=1$ [$\Omega$] の抵抗を用いる.

\section{演習問題: 表の記述}

\begin{table}[htb]
    \centering
    \caption{抵抗$R=1.0$ [$\Omega$] の場合の電流値$I$ [A] と電圧値$E$ [V].}
    \label{table:EVR1}
    \begin{tabular}{cc}
        \hline
        電流$I$ [A] & 電圧$E$ [V]\\\hline\hline
        $1.0$ & $1.0$ \\\hline
        $2.0$ & $2.0$ \\\hline
        $3.0$ & $3.0$ \\\hline
        $4.0$ & $4.0$ \\\hline
        $5.0$ & $5.0$ \\\hline
        $6.0$ & $6.0$ \\\hline
        \end{tabular}
    \end{table}

\end{document}