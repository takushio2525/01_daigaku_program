\documentclass[uplatex]{jsarticle}

\usepackage{amsmath}
\usepackage[dvipdfmx]{graphicx}

\setcounter{tocdepth}{3}
\usepackage{float}
\usepackage{moreverb}
\usepackage{lscape}
%\pagestyle{empty}
%\usepackage{wrapfig}
%\usepackage{url}
%\usepackage{EasyLayout}

\usepackage{ascmac}
%\usepackage{fancybx}

%\pagestyle{myheadings}


\begin{document}


\title{第8回目の課題}
\author{25G1065 塩澤匠生}
%\date{2015年11月13日}
\maketitle


\section{はじめに}
習志野市に10店舗を展開するCIT スーパーは売上高の減少に悩まされている.
そこで,この原因についての調査を行い,最も効率よく売上高減少を防止できる策を考える.


\section{手法}
調査のために,CITスーパーを統括するマネージャーは図\ref{fig:anket}に示すアンケートを作成し,実施した.
接客と品揃えと立地についての満足度を5段階で評価してもらいう形式で実施し,表\ref{table:anket_result}に示すような結果を得られた.
この結果から売上高と相関のある項目を見つけて,その項目を改善することで売上高の減少を防止することを目指す.



\begin{figure}[H]
    \centering
    \includegraphics[width=0.5\textwidth]{anket.png}
    \caption{実施したアンケート}
    \label{fig:anket}
\end{figure}

\begin{table}[H]
    \centering
    \caption{実施したアンケートの結果}
    \label{table:anket_result}
    \begin{tabular}{ccccc}
        \hline
        店舗 & 1. 接客スコア & 2. 品揃えスコア & 3. 立地スコア & 売上高 (億円) \\\\ \hline
        A & 3.42 & 4.03 & 2.22 & 305.29 \\\\ \hline
        B & 4.49 & 4.07 & 1.84 & 281.99 \\\\ \hline
        C & 4.71 & 2.54 & 4.76 & 382.03 \\\\ \hline
        D & 0.39 & 3.11 & 3.62 & 346.11 \\\\ \hline
        E & 2.14 & 1.59 & 2.81 & 322.68 \\\\ \hline
        F & 4.67 & 4.63 & 1.63 & 277.73 \\\\ \hline
        G & 4.92 & 2.90 & 4.68 & 372.33 \\\\ \hline
        H & 4.13 & 2.85 & 2.69 & 314.68 \\\\ \hline
        I & 0.61 & 4.61 & 4.38 & 365.65 \\\\ \hline
        J & 3.78 & 3.10 & 4.28 & 357.50 \\\\ \hline
    \end{tabular}
\end{table}

\section{結果}
アンケートの結果をもとに,売上高と各スコアとの相関を調べた結果,図\ref{fig:correlation}に示すような結果が得られた.
立地スコアと売上高には$r=0.99$の強い相関が見られるということがわかった.


\begin{figure}[H]
    \centering
    \includegraphics[width=1\textwidth]{graph/csv_graph.png}
    \caption{売上高と各項目の相関}
    \label{fig:correlation}
\end{figure}


\section{おわりに}


\end{document}



























