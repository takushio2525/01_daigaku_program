\documentclass[uplatex]{jsarticle}
\usepackage{amsmath}
\usepackage[dvipdfmx]{graphicx}

\setcounter{tocdepth}{3}
\usepackage{float}
\usepackage{moreverb}
\usepackage{lscape}
%\pagestyle{empty}
%\usepackage{wrapfig}
%\usepackage{url}
%\usepackage{EasyLayout}

\usepackage{ascmac}
%\usepackage{fancybx}

%\pagestyle{myheadings}



\begin{document}


\title{二分法とニュートン法の収束の速さについて}
\author{25G1065 塩澤匠生}

%\date{2015年11月13日}
\maketitle


\section{はじめに}
自然界の現象をモデル化するのに方程式が用いられる.
オームの法則や直線運動については線形の方程式でモデル化できるが
落下運動や人口増加モデルについては線形であるとモデル化できないため
非線形方程式のまま扱う必要がある.


 一般の非線形方程式は解析的に解けないため数値計算に頼る必要がある.
非線形方程式の解を数値計算によっても読める手法として中間値の定理によって
解を求める二分法と接線を利用して解を求めるニュートン法というものがある.


 今回はニュートン法と二分法どちらの方が早く解が収束するかを調べる.
調べることでどちらの手法の方が解を早く求められるかが分かる.
解が収束する速さを求める手法としてc言語でのプログラムを用いる.
\section{手法}
% この下の行に入力

\subsection{方程式の説明}

本レポートでは,(\ref{eq:1})式に示される関数$f(x)$の$f(x)=0$の解を2分法とニュートン法により求める.

\begin{equation}
f(x)=\cos x- x^2\label{eq:1}
\end{equation}

\subsection{2分法・ニュートン法の説明}

\subsection{プログラムの仕様}

\subsection{評価指標}



\end{document}



























