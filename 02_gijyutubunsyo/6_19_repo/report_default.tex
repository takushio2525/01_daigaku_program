\documentclass[uplatex]{jsarticle}
\usepackage{amsmath}
\usepackage[dvipdfmx]{graphicx}

\setcounter{tocdepth}{3}
\usepackage{float}
\usepackage{moreverb}
\usepackage{lscape}
%\pagestyle{empty}
%\usepackage{wrapfig}
%\usepackage{url}
%\usepackage{EasyLayout}

\usepackage{ascmac}
%\usepackage{fancybx}

%\pagestyle{myheadings}



\begin{document}


\title{期末試験の最終レポートのテンプレート}
\author{学生番号と名前をかく}
%\date{2015年11月13日}
\maketitle


\section{はじめに}
% この下の行に入力

\section{手法}
% この下の行に入力

\subsection{方程式の説明}

本レポートでは,(\ref{eq:1})式に示される関数$f(x)$の$f(x)=0$の解を2分法とニュートン法により求める.

\begin{equation}
f(x)=\cos x- x^2\label{eq:1}
\end{equation}

\subsection{2分法・ニュートン法の説明}

\subsection{プログラムの仕様}

\subsection{評価指標}



\end{document}



























