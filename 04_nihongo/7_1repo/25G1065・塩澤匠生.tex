\documentclass[uplatex,dvipdfmx]{jsarticle}

\usepackage[uplatex,deluxe]{otf} % UTF
\usepackage[noalphabet]{pxchfon} % must be after otf package
\usepackage{stix2} %欧文&数式フォント
\usepackage[fleqn,tbtags]{mathtools} % 数式関連 (w/ amsmath)
\usepackage{hira-stix} % ヒラギノフォント&STIX2 フォント代替定義(Warning回避)

\usepackage{ascmac}
\usepackage{url}
\usepackage{float}
\usepackage{moreverb}
\usepackage{lscape}

\begin{document}

\title{景気の変化ー2010年から2020年の変化の考察ー}
\author{25G1065 塩澤匠生}
\maketitle

内閣府(政府広報室)は「社会意識に関する世論調査」を毎年実施している。この調査における2010年から
2020年現在の日本の状況について「良い方向に向かっている分野」について、「悪い方向に向かっている分野」
について、「良い方向に向かっている分野」で回答の多かった「教育」と「悪い方向に向かっている分野」
で回答の多かった「景気」という項目について10年間の変化を分析し、考察したい。

「教育」の項目については、2010年に11.8\%、2020年には17.3\%が「良い方向に向かっている分野」
としてあげた。その間、特に変化が目立つのが、前年度から約6ポイント増加した2014年(17.2\%)である。

「景気」の項目については、2010年に63.1\%、2020年には31.5\%が「悪い方向に向かっている分野」としてあげた。
その間、特に変化が目立つのが前年度から約22ポイント減った2013年(36.1\%)である。

まず、「良い方向に向かっている分野」の「教育」が、2014年(17.2\%)に増えた背景
として前年の「いじめ防止対策推進法」施行の影響が考えられる\cite{ijime}。2018年にも幼児教育・保育の無償化があったり、
2020年には私立高校授業料の実質無償化があったがその影響はあまり見られず、2014年以降それほど増減もなくほぼ横ばいである。

一方、「悪い方向に向かっている分野」の「物価」が、2013年(36.1\%)に減ったのは前年の急激な円安(1ドル80円から100円)
が影響していると考えられる\cite{ennyasu}。
円安の影響で海外からの観光客が増え景気が良くなったこと\cite{ennyasu2}で
「悪くなった」という回答が減っているのではないか。

2012年から2015年の景気の変動について考察したい。最も大きく値が変化したのは2012年から2013年にかけてであるが
2012年から2014年で合計約39ポイントも減少している。
2013年に起こった大規模な金融緩和で円安が進行し、物価が上がったことで国民はデフレから脱却して、景気が良くなる
と思ったのではないだろうか。
しかし、2015年には2014年から比べて11.3\%多くの人が
景気が悪い方に向かっていると回答した。2014年から2015年に
起こったことを調べたところ、2014年に消費税が5\%から8\%に
引き上げられたことが原因として考えられる。2013年までで
景気が良くなったことで消費税を導入した結果、
国民が景気が悪くなったと感じる様になってしまったと考えられる。

2012年から2014年では景気が悪くなっていると回答した人は
大きく減少した。円安が進み、海外からの観光客が増え景気が良く
なったことが原因だと考えられる。しかし、昨今円安が更に進行
しているが、ニュースなどを見ていると明らかに景気が悪くなっている
とされている。正直今回調べていて過去に円安で景気が良くなった
ということがあったことにすごく驚いた。なぜ今円安が進行している
のに景気が良くならないのか今後調査していきたい。

\renewcommand{\refname}{注}
\begin{thebibliography}{11}

\bibitem{ijime} 「いじめ防止対策推進法(平成25年法律第71号)」文部科学省、
2025年7月1日閲覧、<\url{https://www.mext.go.jp/a_menu/shotou/seitoshidou/1337278.htm}>


\bibitem{ennyasu} 「株高41年ぶり、円安34年ぶり… 歴史的値動きの1年」日本経済新聞、
2025年7月1日閲覧、<\url{https://www.nikkei.com/article/DGXNASGC30021_Q3A231C1MM8000/}>


\bibitem{ennyasu2} 「2015年を振り返る[コラムvol.283]」公益財団法人日本交通公社、
2025年7月1日閲覧、<\url{https://www.jtb.or.jp/researchers/column/column-lookingbuck-2015/}>


\end{thebibliography}

\end{document}
