\documentclass[uplatex]{jsarticle}
\usepackage{amsmath}
\usepackage[dvipdfmx]{graphicx}

\setcounter{tocdepth}{3}
\usepackage{float}
\usepackage{moreverb}
\usepackage{lscape}
%\pagestyle{empty}
%\usepackage{wrapfig}
%\usepackage{url}
%\usepackage{EasyLayout}

\usepackage{ascmac}
%\usepackage{fancybx}

%\pagestyle{myheadings}
\usepackage{hyperref}



\begin{document}


\title{patternlib仕様書}
\author{25G1065 塩澤匠生}

%\date{2015年11月13日}
\maketitle


\section{概要}
このライブラリは3次元配列に格納されているデータを下に
ターミナルに指定した書式で出力するライブラリである.
3次元配列のデータ形式はarray[n文字目][行][列]という
形式とする.0で空白,1で表示を示すデータとする.
カラー出力する際はANSIエスケープコードの下一桁の数字
を代入するものとする.
行と列の個数はデフォルトで8になっている.COL_SIZEと
ROW_SIZEの値を変更することで任意で調整可能である.

\section{定数の説明}
\subsection{COL_SIZE}
配列で表現する1文字の行数を指定する定数.
\subsection{ROW_SIZE}
配列で表現する1文字の列数を指定する定数.


\end{document}



























